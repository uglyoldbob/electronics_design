\documentclass[letterpaper,12pt,twoside]{book}

% Uncomment the following line to allow the usage of graphics (.png, .jpg)
%\usepackage[pdftex]{graphicx}
% Comment the following line to NOT allow the usage of umlauts
\usepackage[utf8]{inputenc}
\usepackage {tabu}
\usepackage{filecontents}
\usepackage[table]{xcolor}
\usepackage {longtable}
\usepackage{fancyhdr}
\usepackage{arydshln}
\usepackage{tikz-timing}[2009/05/15]

% define lightgray
\definecolor{lightgray}{gray}{0.9}

% alternate rowcolors for all tables
\let\oldtabular\tabular
\let\endoldtabular\endtabular
\renewenvironment{tabular}{\rowcolors{2}{white}{lightgray}\oldtabular}{\endoldtabular}

% alternate rowcolors for all long-tables
\let\oldlongtable\longtable
\let\endoldlongtable\endlongtable
\renewenvironment{longtable}{\rowcolors{2}{white}{lightgray}\oldlongtable} {
\endoldlongtable}

\usepackage{geometry}
\geometry{textwidth=7.5in}

\pagestyle{fancy}
\fancyhf{}
\fancyfoot[CE,CO]{\leftmark}
\fancyfoot[RE,RO]{Electronics Design [User Manual]\thepage}
\renewcommand{\headrulewidth}{2pt}
\renewcommand{\footrulewidth}{1pt}

% Start the document
\begin{document}
\title{Electronics Design Technical Manual}
\author{Thomas Epperson}
\maketitle
\newpage
\tableofcontents
\newpage

\chapter {Use cases}

\chapter {Libraries}
Libraries contain content uses as the building blocks of a design. Libraries are stored in a variety of ways, and the ways to access those libraries depend on how they are stored.

\section {Local File}
A library can be contained in a local file. A local file may be network accessible by means of some standard file mapping technology. Probably not the best option to use for a network accessible library.

\subsection {Format}
The format of a library saved to a plain file is a follows. The file extension is to be determined.
\begin {center}
\begin {tabular} { |c|c|c|  }
\hline
 a & b & c \\
 d & e & f \\
 g & h & i \\
\hline
\end {tabular}
\end {center}


\end{document}
